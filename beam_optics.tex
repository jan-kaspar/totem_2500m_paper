\section{Beam Optics}
\label{sec:beam optics}

The beam optics relate the proton kinematical states at the IP and at the RP location. A proton emerging from the interaction vertex $(x^*$, $y^*)$ at the angle $(\theta_x^*,\theta_y^*)$ (relative to the $z$ axis) and with momentum $p\,(1+\xi)$, where $p$ is the nominal initial-state proton momentum, is transported along the outgoing beam through the LHC magnets. It arrives at the RPs in the transverse position
\begin{equation}
\label{eq:prot trans}
	\begin{aligned}
		x(z_{\rm RP}) =& L_x(z_{\rm RP})\, \theta_x^*\ +\ v_x(z_{\rm RP})\, x^*\ +\ D_x(z_{\rm RP})\, \xi\ ,\cr
		y(z_{\rm RP}) =& L_y(z_{\rm RP})\, \theta_y^*\ +\ v_y(z_{\rm RP})\, y^*\ +\ D_y(z_{\rm RP})\, \xi \quad
	\end{aligned}
\end{equation}
relative to the beam centre. This position is determined by the optical functions, characterising the transport of protons in the beam line and controlled via the LHC magnet currents.
The effective length $L_{x,y}(z)$, the magnification $v_{x,y}(z)$ and the dispersion $D_{x,y}(z)$ quantify the sensitivity of the measured proton position to the scattering angle, the vertex position and the momentum loss, respectively. Note that for elastic collisions the dispersion terms $D\,\xi$ can be ignored because the protons do not lose any momentum. The values of $\xi$ only account for the initial state momentum offset ($\approx 10^{-3}$) and variations ($\approx 10^{-4}$). Due to the collinearity of the two elastically scattered protons and the symmetry of the optics, the impact of $D\,\xi$ on the reconstructed scattering angles is negligible compared to other uncertainties.

% * Dx ~ 3cm, Dy ~ 3mm
% * with energy fluctuation si(xi) ~ 10^-4, one gets si(x) ~ 3um which is much less that the sensor resolution
% * with energy offset si(xi) ~ 10^-3, common for both beams, the impact on scattering angles is L-R antisymmetric, therefore what counts is DxR - DxL ~ 4E-3, thus in x the effect is ~ 4um, again neglibile wrt. the sensor resolution

The data for the analysis presented here have been taken with a new, special optics, the $\beta^{*} = 2500\un{m}$, specifically developed for measuring low-$|t|$ elastic scattering and conventionally labelled by the value of the $\beta$-function at the interaction point. It maximises the vertical effective length $L_{y}$ and minimises the vertical magnification $|v_{y}|$ at the RP position $z = 220\,$m (Table~\ref{tab:optics}). This configuration is called ``parallel-to-point focussing'' because all protons with the same angle in the IP are focussed on one point in the RP at 220\,m. It optimises the sensitivity to the vertical projection of the scattering angle -- and hence to $|t|$ -- while minimising the influence of the vertex position. In the horizontal projection the parallel-to-point focussing condition is not fulfilled, but -- similarly to the $\beta^{*} = 1000\,$m optics used for a previous measurement~\cite{totem-8tev-1km} -- the effective length $L_{x}$ at $z = 220\,$m is sizeable, which reduces the uncertainty in the horizontal component of the scattering angle. The very high value of $\beta^*$ also implies very low beam divergence which is essential for accurate measurement at very low $|t|$.

\begin{table}
\caption{
Optical functions for elastic proton transport for the $\beta^{*} = 2500\,$m optics. The values refer to the right arm, for the left one they are very similar.
}
\label{tab:optics}
\begin{center}
\vskip-3mm
\begin{tabular}{ccccc}\hline
RP unit & $L_x$ & $v_x$ & $L_y$ & $v_y$ \cr\hline
210-fr & $73.05\un{m}$ & $-0.634$ & $244.68\un{m}$ & $+0.009$ \cr
220-fr & $51.10\un{m}$ & $-0.540$ & $282.96\un{m}$ & $-0.018$ \cr
\hline
\end{tabular}
\end{center}
\end{table}
