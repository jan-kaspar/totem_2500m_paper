\section{Summary}
\label{sec:summary}

The measurement of elastic differential cross-section disfavours the purely-exponential low-$|t|$ behaviour at $\sqrt s = 13\un{TeV}$, similarly to the previous observation at $8\un{TeV}$. Thanks to the very low-$|t|$ reach, the first extraction of the $\rho$ parameter at $\sqrt s = 13\un{TeV}$ was made by exploiting the Coulomb-nuclear interference. The fit with conditions similar to past experiments yields $\rho = 0.09 \pm 0.01$, one of the most precise $\rho$ determinations in history. The fit over the maximum of data points and with maximum reasonable flexibility of the fit function gives $\rho = 0.10 \pm 0.01$.

Also thanks to the very low $|t|$ reach, it was possible to apply the ``Coulomb normalisation'' technique for the first time at the LHC and obtain another total cross-section measurement $\sigma_{\rm tot} = (110.3 \pm 3.5)\un{mb}$ completely independent from the previous TOTEM measurement at $\sqrt s = 13\un{TeV}$ \cite{totem-13tev-90m} but well compatible with it. Since these two measurements are independent, it is possible to calculate the weighted average yielding $\sigma_{\rm tot} = (110.5 \pm 2.4)\un{mb}$.

The updated collection of TOTEM's $\sigma_{\rm tot}$ and $\rho$ data presents a stringent test of model descriptions. For an indicative example, none of the models considered by the COMPETE collaboration is compatible with both $\sigma_{\rm tot}$ and $\rho$.

For both models found to be consistent with TOTEM's data, the inclusion of a crossing-odd 3-gluon-state exchange in the $t$-channel was essential for reaching the good agreement with the data.

If it is demonstrated in future that the crossing-odd exchange component is unimportant for elastic scattering, the low $\rho$ value determined in this publication represents the first experimental evidence for slowing down of the total cross-section growth at higher energies, leading to a deviation from most current model expectations.

We observe significant incompatibilities between $\rm pp$ and $\rm p \bar p$ differential cross-section (in the non-perturbative t-range): this implies experimental evidence of crossing-odd exchange in the $t$-channel, hence of 3-gluons (compound) exchange \cite{bartels-2000,bartels-2001}. This in turn implies the existence of vector glueballs, as required by the $s$-$t$ channels duality \cite{veneziano-1968}. As a notable example, lattice QCD builds vector glueball solution as a 3-gluons state with quantum numbers $1^{--}$ \cite{chen-2006}.
