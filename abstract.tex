The TOTEM experiment at the LHC has performed the first measurement at $\sqrt s = 13\un{TeV}$ of the $\rho$ parameter, the real to imaginary ratio of the nuclear elastic scattering amplitude at $t=0$, obtaining the following results: $\rho = 0.09 \pm 0.01$ and $\rho = 0.10 \pm 0.01$, depending on different physics assumptions and mathematical modelling. The unprecedented precision of the $\rho$ measurement, combined with the TOTEM total cross-section measurements in an energy range larger than $10\un{TeV}$ (from $2.76$ to $13\un{TeV}$), has implied the exclusion of all the models classified and published by COMPETE. The $\rho$ results obtained by TOTEM are compatible with the predictions, from alternative theoretical models both in the Regge-like framework and in the QCD framework, of a colourless 3-gluon bound state exchange in the $t$-channel of the proton-proton elastic scattering. On the contrary, if shown that the 3-gluon bound state $t$-channel exchange is not of importance for the description of elastic scattering, the $\rho$ value determined by TOTEM would represent a first evidence of a slowing down of the total cross-section growth at higher energies.
%
The very low-$|t|$ reach allowed also to determine the absolute normalisation using the Coulomb amplitude for the first time at the LHC and obtain a new total proton-proton cross-section measurement $\sigma_{\rm tot} = (110.3 \pm 3.5)\un{mb}$, completely independent from the previous TOTEM determination. Combining the two TOTEM results yields $\sigma_{\rm tot} = (110.5 \pm 2.4)\un{mb}$.
