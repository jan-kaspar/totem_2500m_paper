\section{Differential Cross-Section}
\label{sec:differential cross-section}

The analysis method is very similar to the previously published one \cite{totem-1km}. The only important difference stems from using different RPs for the measurement: unit 210-fr instead of 220-nr as in \cite{totem-1km} which was not anymore equipped with sensors. Due to the optics and beam parameters the unit 210-fr has worse low-$|t|$ acceptance, further deteriorated by the tilt of the unit. Consequently, in order to maintain the low-$|t|$ reach essential for this study, the main analysis (denoted ``2RP'') only uses the 220-fr units. Since the abandon of the 210-nr unit may, in principle, result in worse resolution and background suppression, for control reasons, the traditional analysis with all units (denoted ``4RP'') was pursued, too. In Section~\ref{sec:cross checks} the ``2RP'' and ``4RP'' will be compared showing a very good match. In what follows, the ``2RP'' analysis will be described unless stated otherwise.

Section~\ref{sec:event analysis} covers all aspects related to the reconstruction of a single event. Section~\ref{sec:diff cs} describes the steps of transforming a raw $t$-distribution into the differential cross-section. The $t$-distributions for the two diagonals are analysed separately. After comparison (Section~\ref{sec:cross checks}) they are finally merged (Section~\ref{sec:final data merging}).

%----------------------------------------------------------------------------------------------------
\subsection{Event Analysis}
\label{sec:event analysis}

The event kinematics are determined from the coordinates of track hits in the RPs after proper alignment (see Sec.~\ref{sec:alignment}) using the LHC optics (see Sec.~\ref{sec:optics}).

%------------------------------

\subsubsection{Kinematics Reconstruction}
\label{sec:kinematics}

For each event candidate the scattering angles of both protons (one per arm) are first estimated separately. In the ``2RP'' analysis, these formulae are used:
\begin{equation}
\label{eq:kin 1a}
	\theta^{*\rm L,R}_x = {x\over L_x}\ ,\quad \theta^{*\rm L,R}_y = {y\over L_y}
\end{equation}
where L and R refer to the left and right arm, respectively, and $x$ and $y$ stand for the proton position in the 220-fr unit. This one-arm reconstruction is used for tagging elastic events, where the left and right arm protons are compared.

Once a proton pair has been selected, both arms are used to reconstruct the kinematics of the event
\begin{equation}
\label{eq:kin 2a}
		\theta_x^* = {1\over 2} \left( \theta^{*\rm L}_x + \theta^{*\rm R}_x \right)\ ,\qquad
		\theta_y^* = {1\over 2} \left( \theta^{*\rm L}_y + \theta^{*\rm R}_y \right)\ .
\end{equation}
Thanks to the left-right symmetry of the optics and elastic events, this combination leads to cancellation of the vertex terms (cf.~Eq.~(\ref{eq:prot trans})) and thus to improvement of the angular resolution (see Section \ref{sec:resolution}).

Eventually, the scattering angle, $\theta^*$, and the four-momentum transfer squared, $t$, are calculated:

\begin{equation}
\label{eq:th t}
\theta^* = \sqrt{{\theta_x^*}^2 + {\theta_y^*}^2}\ ,\qquad t = - p^2 ({\theta_x^*}^2 + {\theta_y^*}^2)\ ,
\end{equation}
where $p$ denotes the beam momentum.

%------------------------------

\subsubsection{Alignment}
\label{sec:alignment}

TOTEM's usual three-stage procedure (Section 3.4 in~\cite{totem-ijmp}) for correcting the detector positions and rotation angles has been applied: a beam-based alignment prior to the run followed by two offline methods. First, track-based alignment for relative positions among RPs, and second, alignment with elastic events for absolute position with respect to the beam -- repeated in 20 minutes time intervals to check for possible beam movements.

Exploiting all the methods, the alignment uncertainties have been estimated to $25\un{\mu m}$ (horizontal shift), $100\un{\mu m}$ (vertical shift) and $2\un{m rad}$ (rotation about the beam axis). Propagating them through Eq.~(\ref{eq:kin 2a}) to reconstructed scattering angles yields $0.50\un{\mu rad}$ ($0.35\un{\mu rad}$) for the horizontal (vertical) angle. RP rotations induce a bias in the reconstructed scattering angles:
\begin{equation}
\label{eq:alig rot bias}
	\theta_x^* \rightarrow \theta_x^* + c \theta_y^*\ ,\quad
	\theta_y^* \rightarrow \theta_y^* + d \theta_x^*\ ,
\end{equation}
where the proportionality constants $c$ and $d$ have a mean of 0 and a standard deviations of $0.013$ and $3.9\cdot10^{-4}$, respectively.

\TODO{Mention that y misalignment is mostly top-bottom correlated? Mention the uncorrelated contribution?}



%------------------------------

\subsubsection{Optics}
\label{sec:optics}

It is crucial to know with high precision the LHC beam optics between IP5 and the RPs, i.e. the behaviour of the spectrometer composed of the various magnetic elements. The optics calibration has been applied as described in~\cite{totem-optics}. This method uses RP observables to determine fine corrections to the optical functions presented in Eq.~(\ref{eq:prot trans}).

In each arm, the residual errors induce a bias in the reconstructed scattering angles:
\begin{equation}
\label{eq:opt bias}
	\theta_x^* \rightarrow (1 + b_x)\, \theta_x^*\ ,\qquad
	\theta_y^* \rightarrow (1 + b_y)\, \theta_y^*\ .
\end{equation}
the biases $b_x$ and $b_y$ have uncertainties of $0.17\un{\%}$ and $0.15\un{\%}$, respectively, and a correlation factor of $-0.90$. \TODO{true?:} These estimates include the effects of magnet harmonics.

To evaluate the impact on the $t$-distribution, it is convenient to decompose the correlated biases $b_x$ and $b_y$ into eigenvectors of the covariance matrix:
\begin{equation}
\label{eq:opt bias modes}
\begin{pmatrix} b_x^{\rm L}\cr b_y^{\rm L} \cr b_x^{\rm R}\cr b_y^{\rm R} \end{pmatrix} =
	   \eta_1 \underbrace{\begin{pmatrix} -1.608\cdot10^{-3}\cr +1.473\cdot10^{-3}\cr -1.630\cdot10^{-3}\cr +1.477\cdot10^{-3} \end{pmatrix}}_{\rm mode\ 1}
  \ +\ \eta_2 \underbrace{\begin{pmatrix} -5.157\cdot10^{-4}\cr +2.541\cdot10^{-5}\cr +5.566\cdot10^{-4}\cr +2.746\cdot10^{-5} \end{pmatrix}}_{\rm mode\ 2}
  \ +\ \eta_3 \underbrace{\begin{pmatrix} +3.617\cdot10^{-4}\cr +3.625\cdot10^{-4}\cr +3.006\cdot10^{-4}\cr +3.641\cdot10^{-4} \end{pmatrix}}_{\rm mode\ 3}
\end{equation}
normalised such that the factors $\eta_{1,2,3}$ have unit variance. The fourth eigenmode has a neglibile contribution and therefore is not explicitly mentioned.



%------------------------------

\subsubsection{Resolution}
\label{sec:resolution}

TODO

%----------------------------------------------------------------------------------------------------
\subsection{Differential Cross-Section Reconstruction}
\label{sec:diff cs}

For a given $t$ bin, the differential cross-section is evaluated by selecting and counting elastic events:
\begin{equation}
{\d\sigma\over \d t}(\hbox{bin}) =
	{\cal N}\, {\cal U}({\rm bin})\, {\cal B}\, {1\over \Delta t}
	\sum\limits_{t\, \in\, {\rm bin}} {\cal A}(\theta^*, \theta_y^*)\ {\cal E}(\theta_y^*)
	\ ,
\end{equation}
where $\Delta t$ is the width of the bin, ${\cal N}$ is a normalisation factor and the other symbols stand for various correction factors:
 ${\cal U}$ for unfolding of resolution effects, ${\cal B}$ for background subtraction, ${\cal A}$ for acceptance correction and ${\cal E}$ for detection and reconstruction efficiency.

%-------------------------

\subsubsection{Event Tagging}
\label{sec:tagging}

TODO

%-------------------------

\subsubsection{Background}
\label{sec:background}

TODO

%-------------------------

\subsubsection{Acceptance Correction}
\label{sec:acc corr}

TODO

%-------------------------

\subsubsection{Inefficiency Corrections}
\label{sec:ineff corr}

TODO

%-------------------------

\subsubsection{Unfolding of Resolution Effects}
\label{sec:unfolding}

TODO

%-------------------------

\subsubsection{Normalisation}
\label{sec:normalisation}

TODO

%-------------------------

\subsubsection{Binning}
\label{sec:binning}

TODO

%-------------------------

\subsubsection{Systematic Uncertainties}
\label{sec:systematics}

TODO

%----------------------------------------------------------------------------------------------------

\subsection{Systematic Cross-Checks}
\label{sec:cross checks}

TODO


 %----------------------------------------------------------------------------------------------------
\subsection{Final Data Merging}
\label{sec:final data merging}

TODO
