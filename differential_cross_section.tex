\section{Differential Cross-Section}
\label{sec:differential cross-section}

The analysis method is very similar to the previously published one \cite{totem-1km}. The only important difference stems from using different RPs for the measurement: unit 210-fr instead of 220-nr as in \cite{totem-1km} which was not anymore equipped with sensors. Due to the optics and beam parameters the unit 210-fr has worse low-$|t|$ acceptance, further deteriorated by the tilt of the unit. Consequently, in order to maintain the low-$|t|$ reach essential for this study, the main analysis (denoted ``2RP'') only uses the 220-fr units. Since the abandon of the 210-nr unit may, in principle, result in worse resolution and background suppression, for control reasons, the traditional analysis with all units (denoted ``4RP'') was pursued, too. In Section~\ref{sec:cross checks} the ``2RP'' and ``4RP'' will be compared showing a very good match. In what follows, the ``2RP'' analysis will be described unless stated otherwise.

\TODO{mention that normalisation from a different dataset?}

\TODO{retained only data where both diagonals were working - cancellation of systematics}

Section~\ref{sec:event analysis} covers all aspects related to the reconstruction of a single event. Section~\ref{sec:diff cs} describes the steps of transforming a raw $t$-distribution into the differential cross-section. The $t$-distributions are analysed separately for each LHC fill and each diagonal. After comparison (Section~\ref{sec:cross checks}) they are finally merged (Section~\ref{sec:final data merging}).

%----------------------------------------------------------------------------------------------------
\subsection{Event Analysis}
\label{sec:event analysis}

The event kinematics are determined from the coordinates of track hits in the RPs after proper alignment (see Sec.~\ref{sec:alignment}) using the LHC optics (see Sec.~\ref{sec:optics}).

%------------------------------

\subsubsection{Kinematics Reconstruction}
\label{sec:kinematics}

For each event candidate the scattering angles of both protons (one per arm) are first estimated separately. In the ``2RP'' analysis, these formulae are used:
\begin{equation}
\label{eq:kin 1a}
	\theta^{*\rm L,R}_x = {x\over L_x}\ ,\quad \theta^{*\rm L,R}_y = {y\over L_y}
\end{equation}
where L and R refer to the left and right arm, respectively, and $x$ and $y$ stand for the proton position in the 220-fr unit. This one-arm reconstruction is used for tagging elastic events, where the left and right arm protons are compared.

Once a proton pair has been selected, both arms are used to reconstruct the kinematics of the event
\begin{equation}
\label{eq:kin 2a}
		\theta_x^* = {1\over 2} \left( \theta^{*\rm L}_x + \theta^{*\rm R}_x \right)\ ,\qquad
		\theta_y^* = {1\over 2} \left( \theta^{*\rm L}_y + \theta^{*\rm R}_y \right)\ .
\end{equation}
Thanks to the left-right symmetry of the optics and elastic events, this combination leads to cancellation of the vertex terms (cf.~Eq.~(\ref{eq:prot trans})) and thus to improvement of the angular resolution (see Section \ref{sec:resolution}).

Eventually, the scattering angle, $\theta^*$, and the four-momentum transfer squared, $t$, are calculated:

\begin{equation}
\label{eq:th t}
\theta^* = \sqrt{{\theta_x^*}^2 + {\theta_y^*}^2}\ ,\qquad t = - p^2 ({\theta_x^*}^2 + {\theta_y^*}^2)\ ,
\end{equation}
where $p$ denotes the beam momentum.

In the ``4RP'' analysis, the same reconstruction as in \cite{totem-1km} is used which allows for stronger elastic-selection cuts, see Section~\ref{sec:tagging}.



%------------------------------

\subsubsection{Alignment}
\label{sec:alignment}

TOTEM's usual three-stage procedure (Section 3.4 in~\cite{totem-ijmp}) for correcting the detector positions and rotation angles has been applied: a beam-based alignment prior to the run followed by two offline methods. First, track-based alignment for relative positions among RPs, and second, alignment with elastic events for absolute position with respect to the beam -- repeated in 20 minutes time intervals to check for possible beam movements.

Exploiting all the methods, the alignment uncertainties have been estimated to $25\un{\mu m}$ (horizontal shift), $100\un{\mu m}$ (vertical shift) and $2\un{m rad}$ (rotation about the beam axis). Propagating them through Eq.~(\ref{eq:kin 2a}) to reconstructed scattering angles yields $0.50\un{\mu rad}$ ($0.35\un{\mu rad}$) for the horizontal (vertical) angle. RP rotations induce a bias in the reconstructed scattering angles:
\begin{equation}
\label{eq:alig rot bias}
	\theta_x^* \rightarrow \theta_x^* + c \theta_y^*\ ,\quad
	\theta_y^* \rightarrow \theta_y^* + d \theta_x^*\ ,
\end{equation}
where the proportionality constants $c$ and $d$ have a mean of 0 and a standard deviations of $0.013$ and $3.9\cdot10^{-4}$, respectively.

\TODO{Mention that y misalignment is mostly top-bottom correlated? Mention the uncorrelated contribution?}



%------------------------------

\subsubsection{Optics}
\label{sec:optics}

It is crucial to know with high precision the LHC beam optics between IP5 and the RPs, i.e. the behaviour of the spectrometer composed of the various magnetic elements. The optics calibration has been applied as described in~\cite{totem-optics}. This method uses RP observables to determine fine corrections to the optical functions presented in Eq.~(\ref{eq:prot trans}).

In each arm, the residual errors induce a bias in the reconstructed scattering angles:
\begin{equation}
\label{eq:opt bias}
	\theta_x^* \rightarrow (1 + b_x)\, \theta_x^*\ ,\qquad
	\theta_y^* \rightarrow (1 + b_y)\, \theta_y^*\ .
\end{equation}
the biases $b_x$ and $b_y$ have uncertainties of $0.17\un{\%}$ and $0.15\un{\%}$, respectively, and a correlation factor of $-0.90$. \TODO{true?:} These estimates include the effects of magnet harmonics.

To evaluate the impact on the $t$-distribution, it is convenient to decompose the correlated biases $b_x$ and $b_y$ into eigenvectors of the covariance matrix:
\begin{equation}
\label{eq:opt bias modes}
\begin{pmatrix} b_x^{\rm L}\cr b_y^{\rm L} \cr b_x^{\rm R}\cr b_y^{\rm R} \end{pmatrix} =
	   \eta_1 \underbrace{\begin{pmatrix} -1.608\cdot10^{-3}\cr +1.473\cdot10^{-3}\cr -1.630\cdot10^{-3}\cr +1.477\cdot10^{-3} \end{pmatrix}}_{\rm mode\ 1}
  \ +\ \eta_2 \underbrace{\begin{pmatrix} -5.157\cdot10^{-4}\cr +2.541\cdot10^{-5}\cr +5.566\cdot10^{-4}\cr +2.746\cdot10^{-5} \end{pmatrix}}_{\rm mode\ 2}
  \ +\ \eta_3 \underbrace{\begin{pmatrix} +3.617\cdot10^{-4}\cr +3.625\cdot10^{-4}\cr +3.006\cdot10^{-4}\cr +3.641\cdot10^{-4} \end{pmatrix}}_{\rm mode\ 3}
\end{equation}
normalised such that the factors $\eta_{1,2,3}$ have unit variance. The fourth eigenmode has a negligible contribution and therefore is not explicitly mentioned.



%------------------------------

\subsubsection{Resolution}
\label{sec:resolution}

TODO



%----------------------------------------------------------------------------------------------------
\subsection{Differential Cross-Section Reconstruction}
\label{sec:diff cs}

For a given $t$ bin, the differential cross-section is evaluated by selecting and counting elastic events:
\begin{equation}
{\d\sigma\over \d t}(\hbox{bin}) =
	\mathcal{N}\, \mathcal{U}({\rm bin})\, \mathcal{B}\, {1\over \Delta t}
	\sum\limits_{t\, \in\, {\rm bin}} \mathcal{A}(\theta^*_x, \theta_y^*)\ \mathcal{E}(\theta_y^*)
	\ ,
\end{equation}
where $\Delta t$ is the width of the bin, $\mathcal{N}$ is a normalisation factor and the other symbols stand for various correction factors: $\mathcal{U}$ for unfolding of resolution effects, $\mathcal{B}$ for background subtraction, $\mathcal{A}$ for acceptance correction and $\mathcal{E}$ for detection and reconstruction efficiency.



%-------------------------

\subsubsection{Event Tagging}
\label{sec:tagging}


\begin{table}
\caption{The elastic selection cuts. The superscripts R and L refer to the right and left arm. The right-most column gives a typical RMS of the cut distribution.
}
\label{tab:cuts}
\begin{center}
%\vskip-3mm
\begin{tabular}{ccc}\hline
number & cut & RMS ($\equiv 1\sigma$)\cr\hline
1 & $\theta_x^{*\rm R} - \theta_x^{*\rm L}$				& $14\un{\mu rad}$	\cr
2 & $\theta_y^{*\rm R} - \theta_y^{*\rm L}$				& $0.38\un{\mu rad}$	\cr\hline
\end{tabular}
\end{center}
\end{table}

For the ``2RP'' analysis one may apply the cuts requiring the reconstructed-track collinearity between the left and right arm, see Table~\ref{tab:cuts}. The correlation plots corresponding to these cuts are shown in Figure~\ref{fig:cuts}.

\begin{figure}
\begin{center}
\includegraphics{fig/cut_example.pdf}
\caption$ of the elastic events, the cut threshold is set to $4\un{\sigma}$.

The tagging efficiency has been studied by applying the cuts also at the $5\un{\sigma}$-level. This selection has yielded about $0.1\un{\%}$ more events in every $|t|$-bin. This kind of inefficiency only contributes to a global scale factor, which is irrelevant for this analysis because the normalisation is taken from a different data set (cf. Section~\ref{sec:normalisation}).

In the ``4RP'' analysis, thanks to the additional information from the 210-fr units, more cuts can be applied (cf.~Table~2 in~\cite{epl101-el}). In particular the left-right comparison of the reconstructed horizontal vertex position, $x^*$, and vertical position-angle correlation in each arm. Furthermore, since the single-arm reconstruction can disentangle the contributions from $x^*$ and $\theta^*_x$, the angular resolution is better compared to the ``2RP'' analysis and consequently cut 1 becomes more stringent.

\TODO{which xi values are accepted}




%-------------------------

\subsubsection{Background}
\label{sec:background}

As the RPs were very close to the beam, one may expect an enhanced background from coincidence of beam halo protons hitting detectors in the two arms. Other background sources (pertinent to any elastic analysis) are: central diffraction and pile-up of two single diffraction events.

The background rate (i.e.~impurity of the elastic tagging) is estimated in two steps, both based on distributions of discriminators from Table~\ref{tab:cuts} plotted in various situations, see an example in Figure~\ref{fig:tag bckg integ}. In the first step, diagonal data are studied under several cut combinations. While the central part (signal) remains essentially constant, the tails (background) are strongly suppressed when the number of cuts is increased. In the second step, the background distribution is interpolated from the tails into the signal region. The form of the interpolation is inferred from non-diagonal RP track configurations (\textit{45 bottom -- 56 bottom} or \textit{45 top -- 56 top}), artificially treated like diagonal signatures by inverting the $y$ coordinate sign in the arm 45. These non-diagonal configurations cannot contain any elastic signal and hence consist purely of background which is expected to be similar in the diagonal and non-diagonal configurations. This expectation is supported by the agreement of the tails of the blue solid and dashed curves in the figure. Since the non-diagonal distributions are flat, the comparison of the signal-peak size to the amount of interpolated background yields the estimate $1 - \mathcal{B} = \mathcal{O}(10^{-3})$.

\begin{figure}
\begin{center}
\includegraphics{fig/cut_dist_antidgn_cmp.pdf}
\caption{%
Distributions of discriminator 1, i.e. the difference between the horizontal scattering angle reconstructed from the right and the left arm. Data from LHC fill 5314. Black and red curves: data from diagonal 45 bottom -- 56 top, the different colours correspond to various combinations of the selection cuts (see numbering in Table~\ref{tab:cuts}). Blue and green curves: data from anti-diagonal RP configurations, obtained by inverting track $y$ coordinate in the left arm. The vertical dashed lines represent the boundaries of the signal region ($\pm 4\un{\sigma}$).
}
\label{fig:tag bckg integ}
\end{center}
\end{figure}

\TODO{describe Figure~\ref{fig:tag bckg dist}. Apply some correction?}

\begin{figure}
\begin{center}
\includegraphics{fig/t_dist_antidgn_cmp.pdf}
\caption{%
$|t|$ distributions from different diagonal and anti-diagonal configurations, after all cuts and acceptance correction. Data from LHC fill 5314.
}
\label{fig:tag bckg dist}
\end{center}
\end{figure}

%-------------------------

\subsubsection{Acceptance Correction}
\label{sec:acc corr}

The acceptance of elastic protons is limited mostly by two factors: sensor coverage (relevant for low $|\theta^*_y|$) and LHC beam aperture (at $|\theta^*_y| \approx 100\un{\mu rad}$). Since the 210-fr units are tilted, while the 220-fr not, the thin windows do not overlap perfectly. Therefore, there are phase space regions where protons need to traverse thick walls of 210-fr RP before being detected in 220-fr RP. This induces reduced detection efficiency difficult to determine precisely. Consequently these regions (close to the sensor edge facing the beam) have been excluded from the fiducial region used in the analysis, see the magenta lines in Figure~\ref{fig:acc corr princ}.

\begin{figure}
\begin{center}
\includegraphics{fig/acc_phi_lab.pdf}
\caption{%
Distribution of scattering angle projections $\theta_y^*$ vs.~$\theta_x^*$, data from LHC fill 5317. The upper (lower) part comes from the diagonal 45 bottom -- 56 top (45 top -- 56 bottom). The red horizontal lines represent cuts due to the LHC apertures, the magenta lines cuts due to the RP edges. The dotted circles show contours of constant scattering angle $\theta^*$ as indicated in the middle of the plot. The parts of the contours within acceptance are emphasized in thick black.
}
\label{fig:acc corr princ}
\end{center}
\end{figure}

The correction for the above phase-space limitations includes two contributions -- a geometrical correction $\mathcal{A}_{\rm geom}$ reflecting the fraction of the phase space within the acceptance and a component $\mathcal{A}_{\rm fluct}$ correcting for fluctuations around the acceptance boundaries:
\begin{equation}
\mathcal{A}(\theta^*_x, \theta_y^*) = \mathcal{A}_{\rm geom}(\theta^*)\ \mathcal{A}_{\rm fluct}(\theta^*_x, \theta_y^*)\ .
\end{equation}

The calculation of the geometrical correction $\mathcal{A}_{\rm geom}$ is based on the azimuthal symmetry of elastic scattering, experimentally verified for the data within acceptance. As shown in Figure \ref{fig:acc corr princ}, for a given value of $\theta^*$ the correction is given by:
\begin{equation}
\label{eq:acc geom}
\mathcal{A_{\rm geom}}(\theta^*) = {
	\hbox{full circumference}\over 
	\hbox{arc length within acceptance}
} \ .
\end{equation}

The correction $\mathcal{A}_{\rm fluct}$ is calculated analytically from the probability that any of the two elastic protons leaves the region of acceptance due to the beam divergence. The beam divergence distribution is modelled as a Gaussian with the spread determined by the method described in Section~\ref{sec:resolution}. This contribution is sizeable only close to the acceptance limitations. Data from regions with corrections larger than $2$ are discarded.

The full acceptance correction, $\mathcal{A}$, starts about $12$ in the lowest-$|t|$ bin and decreases smoothly towards about $2.1$ at $|t| = 0.2\un{GeV^2}$. Since a single diagonal cannot cover more than half of the phase space, the minimum value of the correction is $2$.



%-------------------------

\subsubsection{Inefficiency Corrections}
\label{sec:ineff corr}

Since the overall normalisation will be determined from another dataset (see Section~\ref{sec:normalisation}), any inefficiency correction that does not alter the $t$-distribution shape does not need to be considered in this analysis (trigger, data acquisition and pile-up inefficiency discussed in~\cite{epl101-el,prl111}). The remaining inefficiencies are related to the inability of a RP to resolve the elastic proton track.

One such case is when a single RP does not detect and/or reconstruct a proton track, with no correlation to other RPs. This type of inefficiency, $\mathcal{I}_1$, is evaluated within the ``4RP'' analysis by removing the studied RP from the tagging cuts, repeating the event selection and calculating the fraction of recovered events. A typical example is given in Figure~\ref{fig:eff uncorr}, showing that the efficiency decreases gently with the vertical scattering angle. This dependence originates from the fact that protons with larger $|\theta_y^*|$ hit the RPs further from their edge and therefore the potentially created secondary particles have more chance to induce additional signal. Since the RP detectors cannot resolve multiple tracks (non-unique association between ``U'' and ``V'' track candidates), a secondary particle track prevents from using the affected RP in the analysis.

\begin{figure}
\begin{center}
\includegraphics{fig/eff3outof4_fits.pdf}
\caption{%
Single-RP uncorrelated inefficiency for the 220-fr bottom RP in the right arm. The rapid drop at $\theta_y^* \approx 4\un{\mu rad}$ is due to acceptance effects at the sensor edge. The red lines represent a linear fit of the efficiency dependence on the vertical scattering angle (solid) and its extrapolation to the regions affected by acceptance effects (dashed).
}
\label{fig:eff uncorr}
\end{center}
\end{figure}

Another source of inefficiency are proton interactions in a near RP affecting simultaneously the far RP downstream. The contribution from these near-far correlated inefficiencies, $\mathcal{I}_2$, is determined by evaluating the rate of events with high track multiplicity ($\gtrsim$ 5) in both near and far RPs. Events with high track multiplicity simultaneously in a near top and near bottom RP are discarded as such a shower is likely to have started upstream from the RP station and thus be unrelated to the elastic proton interacting with detectors. The outcome, $\mathcal{I}_2 \approx (1.5 \pm 0.7)\un{\%}$, is compatible between left/right arms and top/bottom RP pairs and compares well to Monte-Carlo simulations (e.g.~section 7.5 in \cite{hubert-thesis}).

The full correction is calculated as
\begin{equation}
\label{efficiency}
	\mathcal{E}(\theta_y^*) = {1\over 1 - \left( \sum\limits_{i\in \rm RPs} \mathcal{I}^i_1(\theta_y^*) + 2 \mathcal{I}_2 \right) } \ .
\end{equation}
The first term in the parentheses sums the contributions from the diagonal RPs used in the analysis. In the ``2RP'' analysis it increases from about $6.9$ to $8.5\un{\%}$ from the lowest to the highest $|\theta_y^*|$. For the ``4RP'' analysis, the range is from $10.5$ to $13.0\un{\%}$. The typical uncertainty is about $0.4\un{\%}$.



%-------------------------

\subsubsection{Unfolding of Resolution Effects}
\label{sec:unfolding}

The correction for resolution effects is determined by the following iterative procedure.
\begin{itemize}
\item[1.] The differential cross-section data are fitted by a smooth curve.
\item[2.] The fit is used in a numerical-integration calculation of the smeared $t$-distribution (using the resolution parameters determined in Section~\ref{sec:resolution}). The ratio between the smeared and the non-smeared $t$-distributions gives a set of per-bin correction factors.
\item[3.] The corrections are applied to the observed (yet uncorrected) differential cross-section yielding a better estimate of the true $t$-distribution.
\item[4.] The corrected differential cross-section is fed back to step 1.
\end{itemize}
As the estimate of the true $t$-distribution improves, the difference between the correction factors obtained in two successive iterations decreases. When the difference becomes negligible, the iteration stops. This is typically achieved after the second iteration. 

The final correction $\mathcal{U}$ is significantly different from $1$ only at very low $|t|$ where the rapid cross-section growth occurs, see Figure~\ref{fig:unfolding}.

Several fit parametrisations were tested, however yielding negligible difference in the final correction $\mathcal{U}$, see Figure~\ref{fig:unfolding}.

\begin{figure}
\begin{center}
\includegraphics{fig/unfolding_num_int_model_cmp.pdf}
\caption{%
Unfolding correction as a function of $|t|$. The vertical dashed line indicates the position of the acceptance cut. The two correction curves were obtained with different fit parametrisations used in step 1 (see text).
}
\label{fig:unfolding}
\end{center}
\end{figure}

For the uncertainty estimate, the uncertainties of the $\theta_x^*$ and $\theta_y^*$ resolutions (accommodating the full time variation) as well as fit-model dependence have been considered.
% each contribution giving a few per-mille for the lowest-$|t|$ bin.



%-------------------------

\subsubsection{Normalisation}
\label{sec:normalisation}

TODO

%-------------------------

\subsubsection{Binning}
\label{sec:binning}

TODO

%-------------------------

\subsubsection{Systematic Uncertainties}
\label{sec:systematics}

TODO

%----------------------------------------------------------------------------------------------------

\subsection{Systematic Cross-Checks}
\label{sec:cross checks}

TODO


 %----------------------------------------------------------------------------------------------------
\subsection{Final Data Merging}
\label{sec:final data merging}

TODO
