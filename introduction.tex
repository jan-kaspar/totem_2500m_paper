\section{Introduction}
\label{sec:introduction}

The measurement of the elastic scattering and of the total cross section is traditionally explained by theories that describe the behavior of the scattering amplitude functions as a function of energy and the four momentum transfer $t$.
Dynamical theories in the framework of QCD have been proposed by various authors~\cite{Lukaszuk:1973nt,Bartels:1980pe,Kwiecinski:1980wb}[nicolescu, Bartels, Kwiecinski] proven to describe the phenomena in a precise way.
They proposed to describe the scattering processes between identical particles and between a particle and its antiparticle as exchanges of gluons in the $s$ or $t$ plane and introduced next to the Pomeron exchange the possibility of exchanging an odd under charge exchange virtual particle, the Odderon. 
Measurements of scattering at the ISR center of mass energies later showed differences in the elastic differential cross section in $pp$ and $p\bar{p}$ collisions in the region of the dip~\cite{npb262,RevModPhys.57.563}[Amos and Block, Cahn w. ISR summary].
This prompted speculations that the two types of interactions could not be described with the same mechanism in the asymptotic high energy limit.
 
It has not been possible since the ISR to measure the scattering of $pp$ and $p\bar{p}$ at the same center of mass energy, while the theoretical framework to describe particle scattering with QCD has been actively studied.
In fact with the increase of energy in the center of mass the measurements at the CERN $Sp\bar{p} S$ and at the Tevatron provided new  measurements but only for $p\bar{p}$ interaction.
Now at the LHC, at center of mass energies much larger than ever before, new measurements from $pp$ scattering are available\cite{epl101-tot,prl111,Aad:2014dca,Aaboud:2016ijx} [TOTEM, Atlas] at different center of mass energies.

The high luminosity and stability of the LHC has allowed TOTEM and ATLAS in few dedicated periods to obtain very precise measurements of the elastic and total cross section.
The obtained precision should allow to look for differences foreseen by  the theory possibly signaling a contribution to the cross sections by the Odderon.
The high precision obtained in the measurement of $pp$ elastic scattering has already shown that the differential cross section is best described by a non exponential function~\cite{totem-8tev-90m}[TOTEM non-ex] as postulated already long time ago\cite{Lukaszuk:1973nt,RevModPhys.57.563}[?Nicolescu,Block and Cahn...].
This behaviour according to~\cite{Block:2016jem}[Block2016] is  also visible in measurements at lower center of mass energies where the statistics could not rule out the simple exponential behavior preferred until now to describe the elastic scattering.

This non exponentiality might be already an indication of exchange of more than one particle in the elastic scattering processes.
It is then interesting to perform a detailed study of $pp$ scattering at very small momentum transfers ($t \sim 10^{-4}$) when the Nuclear Cross section is as large as the Coulomb cross section and where it is expected that contributions to the exchange from different processes be larger.
Measuring the interference term is one the most sensitive possibility to disentangle contribution from Pomeron and Odderon and possibly other phenomena.

In the following, after a brief description of the experimental apparatus and the Optics used in the LHC for the special $\beta\,=\,2500m$ run, Section~\ref{sec:data taking} gives details of the data-taking conditions.
 The data analysis and reconstruction of the differential cross-section are described in Section~\ref{sec:differential cross-section}. 
 Section~\ref{sec:coulomb} presents the study of the Coulomb-nuclear interference together with the functional form of the hadronic amplitude.  
The values of $\rho$ and $\sigma_{\rm tot}$ are determined.

