\section{Introduction}
\label{sec:introduction}

The TOTEM experiment at the LHC has measured the differential elastic scattering proton-proton cross-section with respect to the four-momentum transfer squared, $t$, down to $|t| = 8\times10^{-4}\un{GeV^2}$ at centre-of-mass energy $\sqrt s = 13\un{TeV}$ using a special $\beta^* = 2.5\un{km}$ optics. This allowed to access the Coulomb-nuclear interference (CNI) and determine the real-to-imaginary ratio of the hadronic amplitude, $\rho$, as well as the total proton-proton cross-section. The TOTEM measurement of $\rho$ at $13\un{TeV}$ has been realised with an unprecedented precision. % of better than $8% uncertainty.

Measurements of the total proton-proton cross-section and $\rho$ have been published in the literature from the low energy range of $\sqrt s \sim 10\un{GeV}$ up to the LHC energy of $8\un{TeV}$ \cite{pdg}. Such experimental measurements have been parametrised by a large variety of phenomenological models in the last decades, and were analysed and classified by the COMPETE collaboration \cite{compete} based on quantitative criteria.

It is shown in the present paper that none of the above mentioned models can describe simultaneously the TOTEM $\rho$ measurement at $13\un{TeV}$ and the ensemble of the total cross-section measurements by TOTEM ranging from $\sqrt s = 2.76$ to $13\un{TeV}$ \cite{totem-7tev-tot2,totem-8tev-90m,totem-8tev-1km,totem-13tev-90m}. The exclusion of the COMPETE published models is quantitatively demonstrated on the basis of the p-values reported in this work. Such conventional modelling of the low-$|t|$ nuclear elastic scattering is based on various forms of Pomeron exchanges and related even-under-crossing scattering amplitudes.

Sophisticated alternative theoretical models exist both in terms of Regge-like theories \cite{nicolescu-1990} and of modern QCD \cite{braun}: they are capable of predicting or taking into account several effects confirmed or observed at LHC energies \cite{totem-7tev-first,totem-8tev-90m}: the existence of a sharp diffractive dip in the proton-proton elastic t-distribution also at LHC energies, the deviation of the elastic diffractive slope, $B$, from a pure exponential as a function of $t$, the deviation of the elastic diffractive slope $B$ from a linear $\log(s)$ dependence as a function of centre-of-mass energy, the variation of the nuclear phase as a function of $t$, the large-$|t|$ power-law behaviour of the elastic $t$-distribution with no oscillatory behaviour, the growth rate of the total cross-section as a function of $\sqrt s$ at LHC energies. Such theoretical frameworks foresee the possibility of more complex $t$-channel exchanges in the proton-proton elastic interaction, including odd-under-crossing scattering amplitude contributions.

The latter were associated to the concept of the Odderon (the odd-under-crossing counterpart of the Pomeron) invented in the '70s \cite{nicolescu-1973} and later confirmed as an essential QCD prediction \cite{lipatov-1986}, as well as they can be quantified in modern QCD \cite{durham-2015-review} where they are represented (in the most basic form) by the exchange of a colourless 3-gluon bound state in the t-channel. Such state would have naturally $J^{PC}=1^{--}$ quantum numbers and is predicted by lattice QCD with a mass of the order of $3\un{GeV}$ (also referred to as oddball or vector glueball).

Normally the amplitudes related to such additional channels are dominated by those of the Pomeron(s) exchange first, and by the Pomeron-photon exchange then, which explains the reasons why their effect has been difficult to be detected experimentally% [HERA]
. At lower energy% [ISR]
the matter was further complicated by the presence of secondary Regge trajectories influencing the potential observation of differences between the proton-proton and proton-antiproton scattering. At high energy (gluonic-dominated interactions)% [LHCyellow]
, one could investigate for both proton-proton and proton-antiproton scattering the diffractive dip, where the imaginary part of the Pomeron amplitude vanishes, however there are no measurements nor facilities allowing a comparison at the same fixed $\sqrt s$ energy.

The Coulomb-nuclear interference at the LHC is an ideal laboratory to probe the exchange of a virtual odd-gluons bound state, because it selects automatically the required quantum numbers in the $t$-range where the interference terms cannot be neglected with respect to the QED and nuclear amplitudes. The highest sensitivity is reached in the $t$-range where the QED and nuclear amplitudes are of similar magnitude, thus this has been the driving factor in designing the acceptance requirements then achieved via the $2.5\un{km}$ optics of the LHC. $\rho$ being an analytical function of the nuclear phase at $t=0$, it represents a sensitive probe of the interference terms into the evolution of the real and imaginary parts of the nuclear amplitude.

Consequently theoretical models have made sensitive predictions via the evolution of $\rho$ as a function of $\sqrt s$ to quantify the effect of the possible the 3-gluon bound state exchange in the elastic scattering $t$-channel. Currently non-excluded theoretical models systematically require significantly lower $\rho$ values at $13\un{TeV}$ than the predicted Pomeron-only evolution of $\rho$ at $13\un{TeV}$, consistently with the $\rho$ measurement reported in the present work.

The confirmation of this result in additional channels would bring, in addition to the evidence for the existence of the QCD predicted 3-gluon bound state, theoretical consequences such as the generalization of Pomeranchuk theorem (i.e.~the total cross-section of proton-proton and proton-antiproton asymptotically having their ratio converging to 1 rather than their difference converging to 0).
