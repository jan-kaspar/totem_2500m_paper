\section{Introduction}
\label{sec:introduction}

The TOTEM experiment at the LHC has measured the differential elastic proton-proton scattering cross-section as a function of the four-momentum transfer squared, $t$, down to $|t| = 8\times10^{-4}\un{GeV^2}$ at the centre-of-mass energy $\sqrt s = 13\un{TeV}$ using a special $\beta^* = 2.5\un{km}$ optics. This allowed to access the Coulomb-nuclear interference (CNI) and to determine the $\rho$ parameter, the real-to-imaginary ratio of the forward hadronic amplitude, with an unprecedented precision.

Measurements of the total proton-proton cross-section and $\rho$ have been published in the literature from the low energy range of $\sqrt s \sim 10\un{GeV}$ up to the LHC energy of $8\un{TeV}$ \cite{pdg-2016}. Such experimental measurements have been parametrised by a large variety of phenomenological models in the last decades, and were analysed and classified by the COMPETE collaboration \cite{compete}.

It is shown in the present paper that none of the above-mentioned models can describe simultaneously the TOTEM $\rho$ measurement at $13\un{TeV}$ and the ensemble of the total cross-section measurements by TOTEM ranging from $\sqrt s = 2.76$ to $13\un{TeV}$ \cite{totem-7tev-tot2,totem-8tev-90m,totem-8tev-1km,totem-13tev-90m}. The exclusion of the COMPETE published models is quantitatively demonstrated on the basis of the p-values reported in this work. Such conventional modelling of the low-$|t|$ nuclear elastic scattering is based on various forms of Pomeron exchanges and related crossing-even scattering amplitudes (not changing sign under crossing, cf.~Section 4.5~in \cite{barone-predazzi}).

Other theoretical models exist both in terms of Regge-like or axiomatic field theories \cite{nicolescu-1992} and of QCD \cite{bartels-1980,kwiecinski-1980,jaroszewicz-1981} -- they are capable of predicting or taking into account several effects confirmed or observed at LHC energies: the existence of a sharp diffractive dip in the proton-proton elastic t-distribution also at LHC energies \cite{totem-7tev-first}, the deviation of the elastic differential cross-section from a pure exponential \cite{totem-8tev-90m}, the deviation of the elastic diffractive slope, $B$, from a linear $\log(s)$ dependence as a function of the centre-of-mass energy \cite{totem-13tev-90m}, the variation of the nuclear phase as a function of $t$, the large-$|t|$ power-law behaviour of the elastic $t$-distribution with no oscillatory behaviour and the growth rate of the total cross-section as a function of $\sqrt s$ at LHC energies \cite{totem-13tev-90m}. These theoretical frameworks foresee the possibility of more complex $t$-channel exchanges in the proton-proton elastic interaction, including crossing-odd scattering amplitude contributions (changing sign under crossing).

The crossing-odd contributions relevant for high energies (where secondary Reggeons are expected to be negligible \cite{broniowski-2018}) were associated with the concept of the Odderon (the crossing-odd counterpart of the Pomeron \cite{levin-1998}) invented in the '70s \cite{nicolescu-1973,nicolescu-1975} and later confirmed as an essential QCD prediction \cite{bartels-1980,kwiecinski-1980,jaroszewicz-1981,braun,ioffe-2010}. They are quantified in QCD (see e.g.~Refs.~\cite{levin-1990,durham-2018}) where they are represented (in the most basic form) by the exchange of a colourless 3-gluon compound state in the t-channel in the non-perturbative regime ($|t|$ ranging from 0 up to roughly the diffractive dip and bump). Such a state would naturally have $J^{PC}=1^{--}$ quantum numbers and is predicted by lattice QCD with a mass of about $3$ to $4\un{GeV}$ (also referred to as vector glueball) \cite{morningstar-1999} as required by the $s$-$t$ channel duality \cite{veneziano-1968}. For completeness, an exchange of a 3-gluon state may also be crossing even in case the state evolves (collapses) into 2 gluons \cite{bartels-2000,bartels-2001,ewerz}. However hereafter, unless specified differently, we will refer only to crossing-odd 3-gluon exchanges -- the crossing-even exchanges can be included in the Pomeron amplitude.

Experimental searches for a 3-gluon compound state have used various channels. In central production the 3-gluon state emitted by one proton may fuse with a Pomeron (photon) emitted from the other proton (electron/positron) and create a detectable meson system \cite{hera-odderon-2002}. However, such processes are dominated by pomeron-photon (photon-photon) fusion, making the observation of a 3-gluon state difficult. In elastic scattering at low energy \cite{breakstone-85}, the observation of 3-gluon compound state is complicated by the presence of secondary Regge trajectories influencing the potential observation of differences between the proton-proton and proton-antiproton scattering. At high energy (gluonic-dominated interactions) \cite{yellow-report}, one could investigate for both proton-proton and proton-antiproton scattering the diffractive dip, where the imaginary part of the Pomeron amplitude vanishes; however there are no measurements nor facilities allowing a comparison at the same fixed $\sqrt s$ energy.

The Coulomb-nuclear interference at the LHC is an ideal laboratory to probe the exchange of a virtual odd-gluons compound state, because it selects the required quantum numbers in the $t$-range where the interference terms cannot be neglected with respect to the QED and nuclear amplitudes squared. The highest sensitivity is reached in the $t$-range where the QED and nuclear amplitudes are of similar magnitude, thus this has been the driving factor in designing the acceptance requirements then achieved via the $2.5\un{km}$ optics of the LHC. The $\rho$ parameter being an analytical function of the nuclear phase at $t=0$, it represents a sensitive probe of the interference terms into the evolution of the real and imaginary parts of the nuclear amplitude.

Consequently theoretical models have made sensitive predictions via the evolution of $\rho$ as a function of $\sqrt s$ to quantify the effect of the possible 3-gluon compound state exchange in the elastic scattering $t$-channel \cite{nicolescu-2007,durham-2018}. Those, currently non-excluded, theoretical models systematically require significantly lower $\rho$ values at $13\un{TeV}$ than the predicted Pomeron-only evolution of $\rho$ at $13\un{TeV}$, consistently with the $\rho$ measurement reported in the present work.

The confirmation of this result in additional channels would bring, besides the evidence for the existence of the QCD-predicted 3-gluon compound state, theoretical consequences such as the generalization of the Pomeranchuk theorem (i.e.~the total cross-section of proton-proton and proton-antiproton asymptotically having their ratio converging to 1 rather than their difference converging to 0).

On the contrary, if the role of the 3-gluon compound state exchange is shown insignificant, the present TOTEM results at $13\un{TeV}$ would imply by the dispersion relations the first experimental evidence for total cross-section saturation effects at higher energies, eventually deviating from the asymptotic behaviour proposed by many contemporary models (e.g.~the functional saturation of the Froissart compound \cite{froissart-1961}).

The two effects, crossing-odd contribution and cross-section saturation, could both be present without being mutually exclusive.

Besides the extraction of the $\rho$ parameter, the very low $|t|$ elastic scattering can be used to determine the normalisation of the differential cross-section -- a crucial ingredient for measurement of the total cross-section, $\sigma_{\rm tot}$. In its ideal form, the normalisation can be determined as the proportionality constant between the Coulomb cross-section known from QED and the data measured at such low $|t|$ that other than Coulomb cross-section contributions can be neglected. This ``Coulomb normalisation'' technique opens the way to another total cross-section measurement at $\sqrt s = 13\un{TeV}$, completely independent of previous results. This publication presents the first successful application of this method to LHC data.

Section~\ref{sec:exp apparatus} of this article outlines the experimental setup used for the measurement. The properties of the special beam optics are described in Section~\ref{sec:beam optics}. Section~\ref{sec:data taking} gives details of the data-taking conditions. The data analysis and reconstruction of the differential cross-section are described in Section~\ref{sec:differential cross-section}. Section~\ref{sec:rho} presents the extraction of the $\rho$ parameter and $\sigma_{\rm tot}$ from the differential cross-section. Physics implications of these new results are discussed in Section~\ref{sec:discussion}.
