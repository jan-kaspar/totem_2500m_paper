\section{Data Taking}
\label{sec:data taking}

The results reported here are based on data taken in September 2016 during a sequence of dedicated LHC proton fills (5313, 5314, 5317 and 5321) with the special beam properties described in the previous section.

% bunching schemes from RunLog
%   singla_6b_4_0_0_1bpi_6inj (5313), singla_6b_5_0_0_1bpi_6inj (5314 and 5317), singla_6b_5_0_0_1bpi_6inj_alt (5321)

% from RunLog: intensity per beam
%  5313: 1.950000e+11 /4 --> 1.840000e+11 /4 bunches
%  5314: 2.100000e+11 /4 --> 2.200000e+11 /5 bunches
%  5317: 2.200000e+11 /5 --> 2.200000e+11 /5 bunches
%  5321: not noted

The vertical RPs approached the beam centre to only about 3 times the vertical beam width, $\sigma_{y}$, thus roughly to $0.4\un{mm}$. The exceptionally close distance was required in order to reach very low $|t|$ values and was possible due to the low beam intensity in this special beam operation: each beam contained only four or five colliding bunches and one non-colliding bunch, each with about $5\times 10^{10}$ protons.

The horizontal RPs were only needed for the track-based alignment and therefore placed at a safe distance of $8\,\sigma_{x} \approx 5$\,mm, close enough to have an overlap with the vertical RPs.

The collimation strategy applied in the previous measurement \cite{totem-8tev-1km} with carbon primary collimators was first tried, however, this resulted in too high beam halo background. To keep the background under control, a new collimation scheme was developed, with more absorbing tungsten collimators closest to the beam in the vertical plane, in order to minimise the out-scattering of halo particles. As a first step, vertical collimators TCLA scraped the beam down to $2\,\sigma_{y}$, then the collimators were retracted to $2.5\,\sigma_{y}$, thus creating a $0.5\,\sigma_{y}$ gap between the beam edge and the collimator jaws. A similar procedure was performed in the horizontal plane: collimators TCP.C scraped the beam to $3\un{\sigma_{x}}$ and then were retracted to $5.5\un{\sigma_{x}}$, creating a $2.5\un{\sigma_x}$ gap. With the halo strongly suppressed and no collimator producing showers by touching the beam, the RPs at $3\,\sigma_{y}$ were operated in a background-depleted environment for about one hour until the beam-to-collimator gap was refilled by diffusion, as diagnosed by the increasing shower rate (red graph in Figure~\ref{fig:rates_vs_time}). When the background conditions had deteriorated to an unacceptable level, the beam cleaning procedure was repeated, again followed by a quiet data-taking period.

\begin{figure*}
\begin{center}
\includegraphics{fig/rates_vs_time.pdf}
\caption{%
Event rates from run 5321 as a function of time. The blue and green graphs give rates of fully reconstructed events in the two diagonal configurations relevant for elastic scattering. The red graph shows a rate of high-multiplicity events in a single RP (bottom pot in sector 45 and unit 210-fr) where no track can be reconstructed. For the other RPs this rate evolution was similar.
}
\label{fig:rates_vs_time}
\end{center}
\end{figure*}

% RunLog: trigger was RP_2arms, BX
The events collected were triggered by a double-arm proton trigger (coincidence of any RP left of IP5 and any RP right of IP5) or a zero-bias trigger (random bunch crossings) for calibration purposes.

In total, a data sample with an integrated luminosity of about $0.4\,\rm nb^{-1}$ was accumulated in which more than 7 million of elastic event candidates were tagged.
