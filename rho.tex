\section{Determination of $\rho$ and total cross-section}
\label{sec:rho}

The value of the $\rho$ parameter can be extracted from the differential cross-section thanks to the effects of Coulomb-nuclear interference (CNI). Explicit treatment of these effects allows also for a conceptually more accurate determination of the total cross-section. 

Our modelling of the CNI effects is summarised in Section \ref{sec:rho cni}, Sections~\ref{sec:rho anal} and \ref{sec:rho anal norm var} describe data fits and results. In Section~\ref{sec:rho anal} the differential cross-section normalisation is fixed by the $\beta^* = 90\un{m}$ data \cite{totem-13tev-90m} (see Section~\ref{sec:normalisation}). In Section~\ref{sec:rho anal norm var} the normalisation is adjusted or entirely determined from the $\beta^* = 2500\un{m}$ data presented in this publication. This allows for different or even completely independent total cross-section determination with respect to Ref.~\cite{totem-13tev-90m}.



%----------------------------------------------------------------------------------------------------
\subsection{Coulomb-Nuclear Interference}
\label{sec:rho cni}

A detailed overview of different CNI descriptions was given in Ref.~\cite{totem-8tev-1km}, Section 6. Here we briefly summarise the choices used for the presented analysis.

The Coulomb amplitude can be derived from QED. In the one-photon approximation it yields the cross-section
\begin{equation}
\label{eq:coul cs}
	{\d\sigma^{\rm C}\over \d t} = {4\pi\alpha^2\over t^2}\,{\mathcal{F}}^4\ ,
\end{equation}
where $\alpha$ is the fine-structure constant and $\mathcal{F}$ represents an experimentally determined form factor. Several form factor determinations have been considered (by Puckett et al., Arrington et al.~and Borkowski et al., see summary in \cite{elegent}) and no difference in results has been observed.

Motivated by the observed differential cross-section, at low $|t|$ the modulus of the nuclear amplitude is parametrised as
\begin{equation}
\label{eq:nuc mod}
\left | {\cal A}^{\rm N}(t) \right | = \sqrt{s\over\pi} {p\over \hbar c} \sqrt{a} \exp\left( {1\over 2} \sum\limits_{n = 1}^{N_b} b_n\, t^n \right)\ .
\end{equation}
The $b_1$ parameter is responsible for the leading exponential decrease, the other $b_n$ parameters can describe small deviations from the leading behaviour. Since the calculation of CNI may, in principle, involve integrations (e.g.~Eq.~(\ref{eq:int kl})), it is necessary to extend the nuclear amplitude meaningfully to higher $|t|$ values, too. In that region, we fix the amplitude to a function that describes well the dip-bump structure observed in the data. In order to avoid numerical problems, the intermediate $|t|$ region is modelled with a continuous and smooth interpolation between the low and high-$|t|$ parts. It has been checked that altering the high-$|t|$ part within reasonable limits has negligible impact on the results.

Several parametrisations have been considered for the phase of the nuclear amplitude. Since one of the main goals of this analysis is to compare the newly obtained $\rho$ value with those at lower energies, we have focused on parametrisations similar to past analyses. Consequently we have considered phases with slow variation at low $|t|$: constant, Bailly and standard from Ref.~\cite{totem-8tev-1km}. No dependence of the results on this choice was observed and therefore only the constant phase
\begin{equation}
\label{eq:nuc phase con}
\arg {\cal A}^{\rm N}(t) = {\pi\over 2} - \arctan\rho = \hbox{const} \ .
\end{equation}
will be retained in what follows. A more complete exploration including phases leading to a peripheral description of elastic scattering is planned for a forthcoming TOTEM publication.

We have used the most general interference formula available in the literature -- the ``KL'' formula \cite{kl94}:
\begin{equation}
\label{eq:int kl}
	\begin{aligned}
		{\d\sigma\over \d t}^{\rm C+N} =& {\pi (\hbar c)^2 \over s p^2} \left | {\alpha s\over t} {\cal F}^2
			+ {\cal A}^{\rm N}\, \Big[1 - \I\alpha G(t)\Big] \right |^2\ ,\cr
		G(t) =& \int\limits_{-4p^2}^0 \d t'\, \log {t'\over t} {\d\phantom{t'}\over \d t'} {\cal F}^2(t')
			  - \int\limits_{-4p^2}^0 \d t' \left( {{\cal A}^{\rm N}(t') \over {\cal A}^{\rm N}(t)} - 1 \right) { I(t, t')\over 2\pi }\ , \cr
		I(t, t') =& \int_0^{2\pi} \d\phi\ {{\cal F}^2(t'')\over t''}\ ,\cr
		t'' =& t + t' + 2\sqrt{t\, t'} \cos\phi\ ,\cr
	\end{aligned}
\end{equation}
which is numerically almost identical to the formula by Cahn \cite{cahn82} as shown in Ref.~\cite{totem-8tev-1km}. The CNI effects were calculated by the computer code from Ref.~\cite{elegent}.


%----------------------------------------------------------------------------------------------------
\subsection{Data fits with fixed normalisation}
\label{sec:rho anal}

The fits of the data from Table~\ref{tab:data} have been carried out with the standard least-squares method, minimising
\begin{equation}
\label{eq:chi sq A}
	\chi^2 = \Delta^\T \mat V^{-1} \Delta\ ,\quad
	\Delta_i = \left.{\d\sigma\over \d t}\right|_{{\rm bin}\ i} - {\d\sigma^{\rm C+N}\over\d t}\left(t^{\rm rep}_{{\rm bin}\ i}\right)\ ,\quad
	\mat V = \mat V_{\rm stat} + \mat V_{\rm syst}\ ,\
\end{equation}
where $\Delta$ is a vector of differences between the differential cross-section data and a fit function $\d\sigma^{\rm C+N}/\d t$ evaluated at the representative point $t^{\rm rep}$ of each bin~\cite{lafferty94}. The minimisation is repeated several times, and the representative points are updated between iterations. The covariance matrix $\mat V$ has two components. The diagonal of $\mat V_{\rm stat}$ contains the statistical uncertainty squared from Table~\ref{tab:data}, $\mat V_{\rm syst}$ includes all systematic uncertainty contributions except the normalisation, see Eq.~(\ref{eq:covar mat}). For improved fit stability, the normalisation uncertainty is not included in the $\chi^2$ definition. In order to propagate this uncertainty to the fit results, the fit is repeated with the normalisation adjusted by $+5.5\un{\%}$ and $-5.5\un{\%}$. For each fit parameter the mean deviation from the fit result with no normalisation adjustment is taken as the effect of normalisation uncertainty, which is then added quadratically to the uncertainty reported by the fit with no bias.

The complete fit procedure has been validated with a Monte-Carlo study confirming that it has negligible bias. It also indicates the composition of the fit parameter uncertainties. For example, for a fit with $N_b = 1$ using data in the ``coarse binning'' up to $|t| = 0.07\un{GeV^2}$, the $\rho$ uncertainty due to the statistical uncertainties is about $0.004$, due to the systematic uncertainties is about $0.003$ and due to the normalisation uncertainty is about $0.009$.

The fits have been found to have negligible dependence on the binning used (see Section~\ref{sec:binning}), the choice of electromagnetic form factor (see text below Eq.~(\ref{eq:coul cs})), the high-$|t|$ nuclear amplitude (see text below Eq.~(\ref{eq:nuc mod})), the choice of the nuclear amplitude phase (see text above Eq.~(\ref{eq:nuc phase con})), the number of fit iterations and the choice of start parameter values for the $\chi^2$ minimisation.

Since the extracted value of $\rho$ may depend on the assumed fit parametrisation etc., an exploration with various fit configurations has been performed: several degrees of the hadronic modulus polynomial, $N_b = 1, 2, 3$, and different sub-samples of the data, constraining them by a maximal value of $|t|$, $|t|_{\rm max}$. For the latter, two values have been chosen. $|t|_{\rm max} = 0.15\un{GeV^2}$ corresponds to the largest interval before the differential cross-section accelerates its decrease towards the dip. It is the largest interval where application of parametrisation from Eq.~(\ref{eq:nuc mod}) is sensible. The other choice, $|t|_{\rm max} = 0.07\un{GeV^2}$, reflects an interval where purely-exponential ($N_b = 1$) nuclear amplitude is expected to provide a good fit. A summary of the fit results is shown in Table~\ref{tab:rho ref fits}. The fit with $N_b = 1$ on the larger $|t|$ range has bad quality, thus the $\rho$ value is not displayed. This shows that the data are not compatible with a pure exponential, similarly to the previous observation at $\sqrt s = 8\un{TeV}$ \cite{totem-8tev-90m,totem-8tev-1km}. Except for this case, all other fit configurations yield good quality and $\rho$ values constrained to a narrow range.

\begin{table}
\caption{%
Summary of results for various fit configurations (medium binning).
}%
\vskip-5mm
\label{tab:rho ref fits}
\begin{center}
\setlength{\tabcolsep}{5pt}
%\def\arraystretch{0.8}
\begin{tabular}{ccccccccc}
\hline
      & \hbox to10pt{} &\multispan3\hss $|t|_{\rm max} = 0.07\un{GeV^2}$\hss & \hbox to10pt{} & \multispan3\hss $|t|_{\rm max} = 0.15\un{GeV^2}$\hss\cr
$N_b$ && $\chi^2/\hbox{ndf}$ & $\rho$ & $\sigma_{\rm tot}\ung{mb}$ && $\chi^2/\hbox{ndf}$ & $\rho$ & $\sigma_{\rm tot}\ung{mb}$\cr
\hline
\vrule width0pt height10pt
1     && $0.9$ & $0.09\pm0.01$ & $ 111.8 \pm 3.1$  &&     $2.1$ & -              & - \cr
2     && $0.9$ & $0.10\pm0.01$ & $ 111.9 \pm 3.1$  &&     $1.0$ & $0.09\pm0.01$  & $ 111.9 \pm 3.1$\cr
3     && $0.9$ & $0.09\pm0.01$ & $ 111.9 \pm 3.0$  &&     $0.9$ & $0.10\pm0.01$  & $ 112.1 \pm 3.1$\cr
\hline
\end{tabular}
\end{center}
\end{table}

\begin{figure}
\vskip-5mm
\begin{center}
\includegraphics{fig/fit_details_exp3_0p15.pdf}
\vskip-2mm
\caption{%
Details of fit with $N_b = 3$ and $|t|_{\rm max} = 0.15\un{GeV^2}$. The fit parameters read: $a = (648\pm 34)\un{mb/GeV^2}$, $b_1 = (10.64 \pm 0.08)\un{GeV^{-2}}$, $b_2 = (4.1 \pm 1.1)\un{GeV^{-4}}$, $b_3 = (10.3 \pm 4.9)\un{GeV^{-6}}$ and $\rho = 0.10 \pm 0.01$.
}
\label{fig:fit exp3 0.15}
\end{center}
\end{figure}


\begin{figure}
\vskip-5mm
\begin{center}
\includegraphics{fig/fit_details_exp1_0p07.pdf}
\vskip-2mm
\caption{%
Details of fit with $N_b = 1$ and $|t|_{\rm max} = 0.07\un{GeV^2}$. The fit parameters read: $a = (643\pm 35)\un{mb/GeV^2}$, $b_1 = (10.39 \pm 0.03)\un{GeV^{-2}}$ and $\rho = 0.09 \pm 0.01$.
}
\label{fig:fit exp1 0.07}
\end{center}
\end{figure}

The extreme cases in Table~\ref{tab:rho ref fits}, combination $N_b=1$ with $|t|_{\rm max} = 0.07\un{GeV^2}$ and $N_b=3$ with $|t|_{\rm max} = 0.15\un{GeV^2}$ have important meanings. In the latter, the largest possible sample is used and maximum flexibility is given to the fit. In that sense, this fit corresponds to the best $\rho$ determination considered. Also, in this case the fit data include many points where the CNI effects are limited. Consequently, the fit can ``learn'' the trend of the nuclear component and ``impose it'' in the region of strong CNI effects. Conversely, the fit configuration $N_b=1$ with $|t|_{\rm max} = 0.07\un{GeV^2}$ relies uniquely on data with sizeable CNI effects. This complementarity explains why these two cases give the extreme values of $\rho$ in Table~\ref{tab:rho ref fits}. Fit details for these two configurations are shown in Figures~\ref{fig:fit exp3 0.15} and \ref{fig:fit exp1 0.07}.

The fit configuration $N_b=1$ with $|t|_{\rm max} = 0.07\un{GeV^2}$ has another important meaning. Considering the shrinkage of the ``forward-cone'', this $|t|$ range is similar to the one used in the UA4/2 analysis \cite{ua4-rho}. This fact may suggest why UA4/2 could not observe deviations of the differential cross-section from pure exponential: the $|t|$ range was too narrow, as it would be for the present data, had the acceptance stopped at $|t| = 0.07\un{GeV^2}$, see Figure~\ref{fig:fit exp1 0.07}. Beyond the $|t|$ range, this fit combination shares more similarities with the UA4/2 fit (and in general with many other past experiments): purely exponential fit and assumption of constant hadronic phase. Moreover, as shown in Ref.~\cite{totem-8tev-1km}, the ``KL'' interference formula \cite{kl94} used in this report gives for this fit configuration very similar $\rho$ results as the ``SWY'' interference formula \cite{wy68} used in many past data analyses. From this point of view this fit combination corresponds to the most fair comparison to previous $\rho$ determinations and their extrapolations, as e.g.~in Figure \ref{fig:rho_vs_s}. It is worth noting that this fit configuration yields a $\rho$ value incompatible at the level of about $4.7\un{\sigma}$
with the preferred COMPETE model (blue curve in the figure).

\begin{figure*}
\vskip-5mm
\begin{center}
\includegraphics{fig/rho_vs_s.pdf}
\caption{%
Dependence of the $\rho$ parameter on energy. The $\rm pp$ (blue) and $\rm p\bar p$ (green) data are taken from PDG \cite{pdg-2010}. TOTEM measurements are marked in red. The two points at $13\un{TeV}$ correspond to the two selected fit cases discussed in text: the lhs.~point to the combination $N_b = 3$ and $|t|_{\rm max} = 0.15\un{GeV^2}$ while the rhs.~point to $N_b = 1$ and $|t|_{\rm max} = 0.07\un{GeV^2}$.
}
\label{fig:rho_vs_s}
\end{center}
\end{figure*}

Further tests were performed in order to probe the stability of the $\rho$ extraction. Since at higher $|t|$ values the effects of CNI are limited, one may conceive a two-step fit: first, use only the higher $|t|$ data to determine the parameters of the hadronic modulus, cf.~Eq.~(\ref{eq:nuc mod}), and second, optimise only $\rho$ with all the data but the hadronic modulus fixed from the first step. Figure~\ref{fig:fit exp3 0.15} indicates that for the first step one needs to include points down to about $|t| = 0.04\un{GeV^2}$ in order to describe correctly the concavity of the data. Performing the two-step fit with $N_b=3$ and with ansatz $\rho = 0.10$ (or $0.14$) yields, at the end, $\rho = 0.103$ (or $0.116$). Although there is a non-zero $\rho$ difference (CNI effects cannot be fully neglected at higher $|t|$), these results demonstrate the pull of the data towards $\rho \approx 0.10$. A logical counterpart of the procedure just described would be to give the higher-$|t|$ data less weight. In its extreme, where the higher-$|t|$ data are not used at all, this has already been covered by fits with $|t|_{\rm max} = 0.07\un{GeV^2}$ discussed above, also showing the preference for lower $\rho$ values.

\begin{figure*}
\vskip-5mm
\begin{center}
\includegraphics{fig/si_el_rho_solution.pdf}
\vskip-3mm
\caption{%
Constraints to the relation between $\rho$ and $\sigma_{\rm el}$ derived from this data set (red line) and from Ref.~\cite{totem-13tev-90m} (blue line). The solution consistent with both constraints is marked with a green dot.
}
\label{fig:si_el rho sol}
\end{center}
\vskip-2mm
\end{figure*}

Figure \ref{fig:si_el rho sol} illustrates a small correction due to a conceptual improvement in combining the data from this publication and from Ref.~\cite{totem-13tev-90m}. The latter assumes certain values of $\rho$ in order to evaluate cross-section estimates which are in turn used in this analysis (see Section~\ref{sec:normalisation}) to estimate $\rho$. This circular dependence can be resolved by considering simultaneously the $\rho$ dependence of $\sigma_{\rm el}$ in Ref.~\cite{totem-13tev-90m} (blue line) and the $\sigma_{\rm el}$ dependence of $\rho$ determined in this analysis (red line). The latter is done as linear interpolation of $\rho$ values extracted assuming $\sigma_{\rm el} = 30.9$ and $31.1\un{mb}$. The linear dependence is confirmed with Monte-Carlo studies. The solution consistent with both datasets (green dot) brings negligible correction to $\rho$ and $-0.03\un{\%}$ correction to the value of $\sigma_{\rm el}$ published in Ref.~\cite{totem-13tev-90m} for $\rho=0.10$.


For each of the fits presented above, the total cross-section can be derived via the optical theorem:
\begin{equation}
\label{eq:si tot}
\sigma_{\rm tot}^2 = {16\pi\, (\hbar c)^2\over 1 + \rho^2}\, a\ ,
\end{equation}
the results are listed in Table~\ref{tab:rho ref fits}.

%----------------------------------------------------------------------------------------------------
\subsection{Data fits with variable normalisation}
\label{sec:rho anal norm var}

Beyond the determination of the $\rho$ parameter, the very low $|t|$ data offer a normalisation method, too. Suppose that the nuclear amplitude in Eq.~(\ref{eq:int kl}) were negligible, then the normalisation of the differential cross-section could be performed with respect to the Coulomb amplitude, known from QED. While such an extreme situation does not occur within the available dataset, Table~\ref{tab:data}, the lowest $|t|$ points receive large contribution from the Coulomb amplitude and can thus be used for normalisation adjustment or determination. In practice, we extend the fit function in Eq.~(\ref{eq:int kl}) with parameter $\eta$
\begin{equation}
\label{eq:fit fcn eta}
{\d\sigma^{C+N}\over \d t} = \eta\ {\pi (\hbar c)^2 \over s p^2} \left | {\alpha s\over t} {\cal F}^2
			+ {\cal A}^{\rm N}\, \Big[1 - \I\alpha G(t)\Big] \right |^2\ ,
\end{equation}
which represents normalisation adjustments with respect the $\beta^* = 90\un{m}$ result \cite{totem-13tev-90m} (corresponding to $\eta = 1$).

In turn, the normalisation can be determined from the $\beta^* = 90\un{m}$ data (Ref.~\cite{totem-13tev-90m} and Section~\ref{sec:normalisation}), from the $\beta^* = 2500\un{m}$ data (this publication) or their combination. This is formalised in the following three approaches.
\begin{itemize}
\item approach 1: normalisation from $90\un{m}$ data, results presented in the previous section (in particular Table~\ref{tab:rho ref fits}),
\item approach 2: normalisation estimated with $2500\un{m}$ data under the constraint (mean, RMS) from the $90\un{m}$ data,
\item approach 3: normalisation estimated only from $2500\un{m}$ data.
\end{itemize}

Since the Coulomb normalisation is performed at very low $|t|$, the presentation in this section will focus on fits with $N_b = 1$. Fits with $N_b = 3$ were tested, too, without significant changes in the results. For the sake of simplicity, only the medium binning will be used in this section. The previous section has shown that results do not depend on the choice of binning.

Since the nuclear-amplitude component cannot be neglected even at the lowest $|t|$ points of the available dataset, Table~\ref{tab:data}, the normalisation determination must be performed with care. It has been found preferable to make the fits in sequence of three steps, using dedicated and physics-motivated fit configurations for each parameter. The parameters of the nuclear amplitude are determined from a ``golden nuclear $|t|$ range'' where $|t|$ is large enough for CNI effects to be small while $|t|$ is small enough for the $N_b = 1$ parametrisation to be suitable. For example, analysing Eq.~(\ref{eq:int kl}) one finds that CNI effects contribute to the full cross-section less than $1\un{\%}$ for $|t| \gtrsim 0.015\un{GeV^2}$. In the nuclear range, the CNI effects can be ignored (charging the residual effects on systematics), making the fit independent of the interference modelling. The normalisation $\eta$, in contrary, is determined from the lowest $|t|$ points which are the only ones having sensitivity to the Coulomb-amplitude component. The $\rho$ parameter is derived from a $|t|$ range where CNI effects are significant, thus complementing the nuclear range, $|t| \lesssim 0.015\un{GeV^2}$. Note that overlapping $|t|$ ranges are used for determination of $\eta$ and $\rho$.

In detail, approach 2 was implemented via the following sequence of fits:
\begin{itemize}
\item step a (determination of $b_1$): fit over range $0.005 < |t| < 0.07\un{GeV^2}$, CNI effects ignored,
\item step b (determination of $\eta$): fit over range $|t| < 0.0015\un{GeV^2}$, $b_1$ fixed from step a, the overall $\chi^2$ receives an additional term $(\eta - 1)^2/ \sigma_\eta^2$, $\sigma_\eta = 0.055$,  which reflects the constraint from the $\beta^* = 90\un{m}$ data,
\item step c (determination of $\rho$ and $a$): fit over range $|t| < 0.07\un{GeV^2}$, $b_1$ fixed from step a, $\eta$ fixed from step b.
\end{itemize}
The $\rho$ and total cross-section results are listed in Table~\ref{tab:rho si tot summary}. $\eta$ was found to be $1.005$ thus deviating by a fraction of sigma ($\sigma_\eta = 0.055$) from the $\beta^* = 90\un{m}$ normalisation. The fit quality (step c) is good: normalised $\chi^2 = 0.90$ for $77$ degrees of freedom.

Approach 3 was implemented via the following sequence of fits:
\begin{itemize}
\item step a (determination of $\eta a^2$ and $b_1$): fit over range $0.015 < |t| < 0.05\un{GeV^2}$, CNI effects ignored, therefore the fit is only sensitive to the product $\eta a^2$, cf.~Eqs.~(\ref{eq:fit fcn eta}) and (\ref{eq:nuc mod}),
\item step b (determination of $\eta$): fit over range $|t| < 0.0023\un{GeV^2}$, $b_1$ and product $\eta a^2$ fixed from step a; therefore $a$ is also determined in this step,
\item step c (determination of $\rho$): fit over range $|t| < 0.015\un{GeV^2}$, $b_1$ fixed from step a, $\eta$ and $a$ fixed from step b.
\end{itemize}
The $\rho$ and total cross-section results are listed in Table~\ref{tab:rho si tot summary}. $\eta$ was found to be $1.020$ thus deviating by less than half a sigma ($\sigma_\eta$) from the $\beta^* = 90\un{m}$ normalisation. The data description quality with parameters collected from steps a, b and c over the range $|t| < 0.05\un{GeV^2}$ is good: normalised $\chi^2 = 1.2$ for $65$ degrees of freedom.

As a test we tried approach 3 implementation with a single fit over $|t| < 0.05\un{GeV^2}$, where all parameters ($\eta$, $a$, $b_1$ and $\rho$) are free and initialised to the values obtained in the previous paragraph. As anticipated above, such fit might have encountered problems due to non-optimal parameter sensitivities on the available $|t|$ range, however, the results listed in Table~\ref{tab:rho si tot summary} are reasonable. $\eta$ was found to be $1.005$ thus deviating by less than a sigma ($\sigma_\eta$) from the $\beta^* = 90\un{m}$ normalisation. The fit quality is good: normalised $\chi^2 = 0.90$ for $61$ degrees of freedom.

The uncertainties for the fits presented above were determined with the following procedure. The experimentally determined $\d\sigma/\d t$ histogram was modified by adding randomly generated fluctuations reflecting the statistical, systematic and normalisation uncertainties (see Section~\ref{sec:systematics}). This was done $100$ times with different random seeds. Each of the modified histograms was fitted by the above sequences, yielding fit parameter samples to determine the parameter fluctuations, i.e.~uncertainties. Histogram modifications resulting in excessive parameter deviations from the unmodified fit ($\Delta\rho > 0.05$ or $\Delta\sigma_{\rm tot} > 10\un{mb}$) were disregarded since such cases would not be accepted in the analysis. This estimation method gives consistent results with Section~\ref{sec:rho anal} (for approach 1) and $\chi^2$-based estimate (from approach 3, single fit). The $\rho$ and $\sigma_{\rm tot}$ uncertainties were cross-checked and adjusted by varying one of the variables with its uncertainty at a time for the steps where several variables were determined.

Table~\ref{tab:rho si tot summary} compares $\rho$ and total cross-section results from Ref.~\cite{totem-13tev-90m} and the approaches described above. All the results are consistent within the estimated uncertainties. The top two rows use the same normalisation, which is a decisive component for the total cross-section value. The larger $\sigma_{\rm tot}$ obtained in this publication can be attributed to the methodological difference: the destructive Coulomb-nuclear interference is explicitly subtracted here. The $\sigma_{\rm tot}$ determinations from Ref.~\cite{totem-13tev-90m} and approach 3 are completely independent, both in terms of data and method, and can therefore be combined for uncertainty reduction. The weighted average yields:
\begin{equation}
\label{eq:si tot comb}
\sigma_{\rm tot} = (110.5 \pm 2.4)\un{mb}\ ,
\end{equation}
which corresponds to $2.2\un{\%}$ relative uncertainty.

\begin{table}
\caption{%
Summary of $\rho$ and total cross-section results.
}%
\vskip-5mm
\label{tab:rho si tot summary}
\begin{center}
\setlength{\tabcolsep}{5pt}
%\def\arraystretch{0.8}
\begin{tabular}{cccc}
\hline
data & method														& $\rho$				& $\sigma_{\rm tot}\ung{mb}$ \cr
\hline
$\beta^* = 90\un{m}$			& Ref.~\cite{totem-13tev-90m}		& -						& $110.6 \pm 3.4$		\cr
\hline
$\beta^* = 2500\un{m}$			& \hbox{approach 1}					& $0.09 \pm 0.01$		& $111.8 \pm 3.2$	\cr
								& \hbox{approach 2}					& $0.09 \pm 0.01$		& $111.3 \pm 3.2$	\cr
								& \hbox{approach 3}					& $0.08(5) \pm 0.01$	& $110.3 \pm 3.5$	\cr
								& \hbox{approach 3 (single fit)}	& $0.10 \pm 0.01$		& $109.3 \pm 3.5$	\cr
\hline
$\beta^* = 90$ and $2500\un{m}$	& Ref.~\cite{totem-13tev-90m} $\oplus$ \hbox{approach 3} &	& $110.5 \pm 2.4$ \cr
\hline
\end{tabular}
\end{center}
\end{table}

Figure~\ref{fig:si tot inel el} compares selected total cross-section measurements at $\sqrt s = 13\un{TeV}$ with past measurements.

\begin{figure*}
\vskip-5mm
\begin{center}
\includegraphics{fig/sigma_tot_el_inel_vs_s.pdf}
\vskip-3mm
\caption{%
Total (red), inelastic (blue) and elastic (green) cross-section as a function of energy, $\sqrt s$. The data are taken from Ref.~\cite{totem-13tev-90m} (and references therein) and Table~\ref{tab:rho si tot summary}. At $\sqrt s = 13\un{TeV}$, three total cross-section points are shown: left hollow corresponds to approach 3, right hollow to Ref.~\cite{totem-13tev-90m} and central filled to the average in Eq.~(\ref{eq:si tot comb}).
}
\label{fig:si tot inel el}
\end{center}
\vskip-3mm
\end{figure*}
